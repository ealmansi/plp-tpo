\documentclass[a4paper, 10pt, twoside]{article}

\usepackage[top=1in, bottom=1in, left=1in, right=1in]{geometry}
\usepackage[utf8x]{inputenc}
\usepackage[spanish, es-ucroman, es-noquoting]{babel}
\usepackage{setspace}
\usepackage{fancyhdr}
\usepackage{lastpage}
\usepackage{amsmath}
\usepackage{amsfonts}
\usepackage{amsthm}
\usepackage{graphicx}
\usepackage{float}
\usepackage{enumitem} % Provee macro \setlist
\usepackage{tabularx}
\usepackage{multirow}
\usepackage{hyperref}
\usepackage{multicol}
\usepackage{verbatim}
\usepackage{listings}
\usepackage[toc, page]{appendix}
\usepackage{color}
\usepackage{alltt}


%%%%%%%%%% Configuración de Fancyhdr - Inicio %%%%%%%%%%
\pagestyle{fancy}
\thispagestyle{fancy}
\lhead{Trabajo Práctico 3 · Paradigmas de Lenguajes de Programación}
\rhead{Aisemberg · Almansi · Levy Alfie}
\renewcommand{\footrulewidth}{0.4pt}
\cfoot{\thepage /\pageref{LastPage}}

\fancypagestyle{caratula} {
   \fancyhf{}
   \cfoot{\thepage /\pageref{LastPage}}
   \renewcommand{\headrulewidth}{0pt}
   \renewcommand{\footrulewidth}{0pt}
}
%%%%%%%%%% Configuración de Fancyhdr - Fin %%%%%%%%%%


%%%%%%%%%% Miscelánea - Inicio %%%%%%%%%%
% Evita que el documento se estire verticalmente para ocupar el espacio vacío
% en cada página.
\raggedbottom

% Deshabilita sangría en la primer línea de un párrafo.
\setlength{\parindent}{0em}

% Separación entre párrafos.
\setlength{\parskip}{0.5em}

% Separación entre elementos de listas.
\setlist{itemsep=0.5em}

% Asigna la traducción de la palabra 'Appendices'.
\renewcommand{\appendixtocname}{Apéndices}
\renewcommand{\appendixpagename}{Apéndices}
%%%%%%%%%% Miscelánea - Fin %%%%%%%%%%

\begin{document}

%%%%%%%%%%%%%%%%%%%%%%%%%%%%%%%%%%%%%%%%%%%%%%%%%%%%%%%%%%%%%%%%%%%%%%%%%%%%%%%
%% Carátula                                                                  %%
%%%%%%%%%%%%%%%%%%%%%%%%%%%%%%%%%%%%%%%%%%%%%%%%%%%%%%%%%%%%%%%%%%%%%%%%%%%%%%%


\thispagestyle{caratula}

\begin{center}

\includegraphics[height=2cm]{DC.png} 
\hfill
\includegraphics[height=2cm]{UBA.jpg} 

\vspace{2cm}

Departamento de Computación,\\
Facultad de Ciencias Exactas y Naturales,\\
Universidad de Buenos Aires

\vspace{4cm}

\begin{Huge}
Trabajo Práctico 3: \\
Programación en Objetos
\end{Huge}

\vspace{0.5cm}

\begin{Large}
Paradigmas de Lenguajes de Programación
\end{Large}

\vspace{1cm}

Segundo Cuatrimestre de 2014

\vspace{4cm}

\begin{tabular}{|c|c|c|}
\hline
Apellido y Nombre & LU & E-mail\\
\hline
Aisemberg, Dan & 242/12 & dea4493@hotmail.com \\
Almansi, Emilio & 674/12 & ealmansi@gmail.com \\
Levy Alfie, Jon\'as & 081/12 & jonaslevy5@gmail.com \\
\hline
\end{tabular}

\end{center}

\newpage

%%%%%%%%%%%%%%%%%%%%%%%%%%%%%%%%%%%%%%%%%%%%%%%%%%%%%%%%%%%%%%%%%%%%%%%%%%%%%%%
%% Índice                                                                    %%
%%%%%%%%%%%%%%%%%%%%%%%%%%%%%%%%%%%%%%%%%%%%%%%%%%%%%%%%%%%%%%%%%%%%%%%%%%%%%%%


\tableofcontents

\newpage


%%%%%%%%%%%%%%%%%%%%%%%%%%%%%%%%%%%%%%%%%%%%%%%%%%%%%%%%%%%%%%%%%%%%%%%%%%%%%%%
%%                                                                           %%
%%%%%%%%%%%%%%%%%%%%%%%%%%%%%%%%%%%%%%%%%%%%%%%%%%%%%%%%%%%%%%%%%%%%%%%%%%%%%%%

\section{Observaciones}

En la siguiente sección, incluimos el código fuente desarrollado. Para obtener una versión en texto plano del código exportamos el mismo mediante el comando \textbf{File Out} de \emph{Pharo}, y posteriormente lo editamos manualmente para mejorar su legibilidad.

Por cada clase que forma parte de la solución, presentamos los métodos de instancia y los métodos de clase asociados. Los métodos de instancia se preceden por un encabezado con el siguiente formato:

\begin{verbatim}
    A subclass: #B
\end{verbatim}

indicando la definición de la clase \emph{B} como subclase de la clase \emph{A}. Después de los métodos de instancia, sigue un encabezado con este otro formato:

\begin{verbatim}
    B class
\end{verbatim}

precediendo a los métodos de clase de la clase \emph{B}.

\section{Código fuente}

\subsection{Eleccion}
\verbatiminput{codigo/eleccion.txt}

    \subsubsection{Lagarto}
    \verbatiminput{codigo/lagarto.txt}

    \subsubsection{Papel}
    \verbatiminput{codigo/papel.txt}

    \subsubsection{Piedra}
    \verbatiminput{codigo/piedra.txt}

    \subsubsection{Spock}
    \verbatiminput{codigo/spock.txt}

    \subsubsection{Tijera}
    \verbatiminput{codigo/tijera.txt}

\subsection{GeneradorRandomParaTest}
\verbatiminput{codigo/generadorrandomparatest.txt}

\subsection{Juego}
\verbatiminput{codigo/juego.txt}

\subsection{Jugador}
\verbatiminput{codigo/jugador.txt}

    \subsubsection{JugadorAdaptativo}
    \verbatiminput{codigo/jugadoradaptativo.txt}

    \subsubsection{JugadorAleatorio}
    \verbatiminput{codigo/jugadoraleatorio.txt}

    \subsubsection{JugadorHumano}
    \verbatiminput{codigo/jugadorhumano.txt}

    \subsubsection{JugadorSiempre}
    \verbatiminput{codigo/jugadorsiempre.txt}

\subsection{Resultado}
\verbatiminput{codigo/resultado.txt}

    \subsubsection{Empate}
    \verbatiminput{codigo/empate.txt}

    \subsubsection{Victoria}
    \verbatiminput{codigo/victoria.txt}

\subsection{TP3Tests}
\verbatiminput{codigo/tp3tests.txt}


\end{document}
